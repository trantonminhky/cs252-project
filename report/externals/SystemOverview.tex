\section{System Overview}
\subsection{Core Functions}
\subsubsection{Granular Search Service}

The search pipeline is built upon a hybrid architecture that combines semantic search and lexical search techniques to optimize both relevance and precision in results. The key components of this pipeline are as follows:

% Changed [htbp] to [H] to force exact placement
\begin{figure}[H]
  \centering
  \includegraphics[width=1.0\linewidth]{externals/uml/ImageSearchPipeline.png}
\end{figure}

% As detailed in Figure \ref{fig:image_pipeline}, the visual search subsystem (Heritage Fusion) operates in three distinct phases:
% \begin{enumerate}
%     \item \textbf{Offline Training:} Public datasets (WikiArt, AHE) are utilized to fine-tune a CLIP ViT-B-16 model via contrastive loss training to better grasp artistic context.
%     \item \textbf{Ingestion Pipeline:} A sequential, memory-optimized GPU process extracts features using a frozen DINOv2 encoder (for style) and the fine-tuned CLIP model (for meaning), while RAM++ generates text tags.
%     \item \textbf{Online Retrieval:} User queries utilize a hybrid fusion approach within a Qdrant Vector DB, calculating a weighted sum of Style and Meaning scores to retrieve and rank the top 5 most relevant results.
% \end{enumerate}

% Changed [htbp] to [H] to force exact placement
\begin{figure}[H]
  \centering
  \includegraphics[width=0.8\linewidth]{externals/uml/GranularSearchPipeline.png}
\end{figure}

\subsubsection{Community Event Service}
This module functions as a dynamic bridge between local culture providers and users, fostering active participation rather than passive observation.

\begin{itemize}
    \item \textbf{Passive Discovery Engine:} The system automatically surfaces active festivals and events by correlating temporal data with location-specific tags. For example, a user browsing a specific pagoda will be notified of upcoming religious rites relevant to that location's history.
    \item \textbf{Decentralized Event Hosting:} Cultural organizations and community leaders are provided with a "Self-hosting" portal. This allows them to create, manage, and broadcast community-driven events directly to interested users, bypassing traditional media gatekeepers.
    \item \textbf{Cultural Differentiation:} Unlike standard event aggregators, this service focuses exclusively on participatory cultural activities, distinguishing the platform as a hub for deep community engagement rather than general entertainment.
\end{itemize}
\subsubsection{Community Event Service}
This module functions as a dynamic bridge between local culture providers and users, fostering active participation rather than passive observation.

\begin{itemize}
    \item \textbf{Passive Discovery Engine:} The system automatically surfaces active festivals and events by correlating temporal data with location-specific tags. For example, a user browsing a specific pagoda will be notified of upcoming religious rites relevant to that location's history.
    \item \textbf{Decentralized Event Hosting:} Cultural organizations and community leaders are provided with a "Self-hosting" portal. This allows them to create, manage, and broadcast community-driven events directly to interested users, bypassing traditional media gatekeepers.
    \item \textbf{Cultural Differentiation:} Unlike standard event aggregators, this service focuses exclusively on participatory cultural activities, distinguishing the platform as a hub for deep community engagement rather than general entertainment.
\end{itemize}

\subsubsection{Recommendation System}
The recommendation engine employs a hybrid filtering approach to personalize the user experience, transforming raw interaction data into curated cultural journeys.

\begin{itemize}
    \item \textbf{Explicit \& Implicit Feedback Loops:} The system captures explicit signals (likes, dislikes) and implicit behaviors (visit history, dwell time, view counts). These inputs are used to build a dynamic user preference profile.
    \item \textbf{Preference Clustering:} By analyzing frequently selected tags (e.g., "Gothic Architecture," "Cham Culture") and visited locations, the algorithm identifies user cohorts with similar tastes. This allows for collaborative filtering, where a user is recommended niche locations that were highly rated by others in their "taste cluster," facilitating serendipitous discovery of new cultural sites.
\end{itemize}
\subsection{Innovation Highlights}
\textbf{Unified Multimodal Search Interface:} Unlike traditional travel platforms restricted to rigid keyword inputs, our system enables a seamless search-by-anything experience. Users can discover locations using natural language prompts, visual inputs, categorical filters, or direct entity names. This flexibility bridges the gap between vague user intent and specific cultural data.

% --- Example Table ---
\begin{table}[htbp]
\centering
\caption{Examples of Multimodal Search Capabilities}
\label{tab:query_examples}
\renewcommand{\arraystretch}{1.3} % Adds padding to rows
\begin{tabular}{|p{3.5cm}|p{10cm}|}
\hline
\textbf{Search Modality} & \textbf{Example User Inputs} \\
\hline
\textbf{Natural Language} & "Quiet places in Saigon to read a book with colonial architecture" \newline "Where can I see traditional water puppet shows?" \\
\hline
\textbf{Visual Input} & \textit{[User uploads photo of Notre Dame Cathedral]} $\rightarrow$ System finds similar Gothic/Romanesque churches. \\
\hline
\textbf{Categorical Filters} & \texttt{\{ "arch\_style": "French Colonial", "district": "1" \}} \newline \texttt{\{ "religion": "Buddhism", "active\_worship": true \}} \\
\hline
\textbf{Direct Entity Lookup} & "Ben Thanh Market" \newline "Ho Chi Minh City Fine Arts Museum" \\
\hline
\end{tabular}
\end{table}
