\section{System Overview}
\subsection{Main Stakeholders}

\begin{itemize}
    \item \textbf{Authenticity Seekers:} These are users driven by a desire for "backstage" elements of culture—local rituals, niche architecture, and specific ethnic history—that are often obscured by commercialized tourism. They struggle to find these experiences through generic search engines, which prioritize high-traffic, commercial venues over cultural depth.

    \item \textbf{International Visitors \& Expatriates:} This demographic often finds online travel information overwhelmed by SEO-optimized generic lists. They perceive existing recommendations as repetitive and lacking in local nuance, making it difficult to distinguish between genuine cultural spots and tourist traps.
\end{itemize}

\textbf{Foreigners:} Find online information
about cultural spots overwhelming,
generic, and repetitive.
\subsection{Core Functions}
\subsubsection{Granular Search Service}

The search pipeline is built upon a hybrid architecture that combines semantic search and lexical search techniques to optimize both relevance and precision in results. The key components of this pipeline are as follows:

\begin{enumerate}
  \item \textbf{Data Embedding and Indexing:} The system utilizes \textbf{ChromaDB} as a vector database. To enhance retrieval accuracy, raw data from architecture and restaurant CSVs is pre-processed into descriptively refined documents before being embedded. We utilize the state-of-the-art \texttt{BAAI/bge-m3} sentence transformer model to generate these vector embeddings.

  \item \textbf{Hybrid Retrieval:} The retrieval phase fuses two distinct methodologies.
        \begin{itemize}
          \item \textit{Semantic Search:} Calculates \textbf{cosine similarity} between the user's query vector and stored document vectors to find the top 10 conceptually relevant results.
          \item \textit{Lexical Search:} Utilizes the \textbf{BM25} algorithm to identify the top 10 results based on exact keyword matching.
        \end{itemize}
        These results are combined using a weighted fusion mechanism with an alpha ratio ($\alpha$) of $0.65$. This weighting prioritizes semantic search (understanding user intent) while retaining the precision of keyword matching for specific entity names.

  \item \textbf{Reranking:} The combined pool of candidates undergoes a final filtering stage using the \texttt{BAAI/bge-reranker-v2-m3} model. This cross-encoder assesses the relevance of each candidate pair deeply, adding precision and selecting the final top 3 results to be returned to the client.
\end{enumerate}
\begin{figure}[htbp]
  \centering
  \includegraphics[width=0.8\linewidth]{externals/uml/GranularSearchPipeline.png}
\end{figure}

\subsubsection{Community Event Service}
\textbf{Passive Discovery:}
Automatically surfaces
active festivals based
on the location's
specific tags, history.


\textbf{Self-hosting:}
Organizations can host
cultural community
events.

\begin{itemize}
    \item \textbf{Explicit \& Implicit Feedback Loops:} The system captures explicit signals (likes, dislikes) and implicit behaviors (visit history, dwell time, view counts). These inputs are used to build a dynamic user preference profile.
    \item \textbf{Preference Clustering:} By analyzing frequently selected tags (e.g., "Gothic Architecture," "Cham Culture") and visited locations, the algorithm identifies user cohorts with similar tastes. This allows for collaborative filtering, where a user is recommended niche locations that were highly rated by others in their "taste cluster," facilitating serendipitous discovery of new cultural sites.
\end{itemize}
\subsection{Innovation Highlights}
\textbf{Unified Multimodal Search Interface:} Unlike traditional travel platforms restricted to rigid keyword inputs, our system enables a seamless search-by-anything experience. Users can discover locations using natural language prompts, visual inputs, categorical filters, or direct entity names. This flexibility bridges the gap between vague user intent and specific cultural data.

% --- Example Table ---
\begin{table}[htbp]
\centering
\caption{Examples of Multimodal Search Capabilities}
\label{tab:query_examples}
\renewcommand{\arraystretch}{1.3} % Adds padding to rows
\begin{tabular}{|p{3.5cm}|p{10cm}|}
\hline
\textbf{Search Modality} & \textbf{Example User Inputs} \\
\hline
\textbf{Natural Language} & "Quiet places in Saigon to read a book with colonial architecture" \newline "Where can I see traditional water puppet shows?" \\
\hline
\textbf{Visual Input} & \textit{[User uploads photo of Notre Dame Cathedral]} $\rightarrow$ System finds similar Gothic/Romanesque churches. \\
\hline
\textbf{Categorical Filters} & \texttt{\{ "arch\_style": "French Colonial", "district": "1" \}} \newline \texttt{\{ "religion": "Buddhism", "active\_worship": true \}} \\
\hline
\textbf{Direct Entity Lookup} & "Ben Thanh Market" \newline "Ho Chi Minh City Fine Arts Museum" \\
\hline
\end{tabular}
\end{table}
